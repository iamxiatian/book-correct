\chapter{写作规范}


\section{公文写作要求}



常见的标点符号使用错误:



\subsection*{1 多个书名号或引号并列时使用顿号分隔}

例1:

小学一年级到二年级必读书目有《安徒生童话》、《格林童话》、《一千零一夜》,也可以看一些带拼音的《三国演义》、《水浒传》、《西游记》等。(错误)

小学一年级到二年级必读书目有《安徒生童话》《格林童话》《一千零一夜》,也可以看一些带拼音的《三国演义》《水浒传》《西游记》等。(正确)

例2:

公安部门要切实加强校园“警务室”、“护学岗”、“安全网”建设,落实护校制度。(错误)

\subsection*{2 在标示数值和起止年限时使用连接号不规范}

例1:

陈某某大学本科四年(2004-2008年),毕业后成功就业。(错误)

陈某某大学本科四年(2004—2008),毕业后成功就业。(正确)

例2:

要加快工程进度,确保科技园3-5年内建成。(错误)

要加快工程进度,确保科技园3~5年内建成。(正确)

解析:标示时间、地域的起止一般用一字线(占一个字符位置),标示数值范围起止一般用浪纹线。

\subsection*{3 在并列分句中使用句号后再使用分号}

例:

一是养老保险安置。对进入企业工作的失地农民要同企业员工一样纳入企业职工基本养老保险;二是医疗保险安置。城镇居民医疗保险制度已建立,可参加城镇居民医疗保险。(错误)

一是养老保险安置。对进入企业工作的失地农民要同企业员工一样纳入企业职工基本养老保险。二是医疗保险安置。城镇居民医疗保险制度已建立,可参加城镇居民医疗保险。(正确)

\subsection*{4 同一形式的括号套用}

例:

围绕政府半年工作开展回头看,认真总结上半年工作,科学谋划下半年工作。(责任单位:各镇(街道))(错误)

围绕政府半年工作开展回头看,认真总结上半年工作,科学谋划下半年工作。[责任单位:各镇(街道)](正确)

\subsection*{5 阿拉伯数字表示次序时使用点号不当}

例1:

1、督促主办单位按时办结。(错误)

1.督促主办单位按时办结。(正确)

例2:

(1)、督促协办单位按时办结。(错误)

(1)督促协办单位按时办结。(正确)

\subsection*{6 在图、表说明文字末尾使用句号}

例:

注:以上各项数据统计截止时间为2012年12月31日;城市人口指常住户籍人口;规模工业企业个数统计为新口径。(错误)

注:以上各项数据统计截止时间为2012年12月31日;城市人口指常住户籍人口;规模工业企业个数统计为新口径(正确)

解析:图或表的短语式说明文字,中间可用逗号,但末尾不用句号。即使有时说明文字较长,前面的语段已出现句号,最后结尾处仍不用句号。

\subsection*{7 在标示发文年号时使用括号不规范}

例:

根据×发[2013]3号文件精神,……(错误)

根据×发〔2013〕3号文件精神,……(正确)

解析:标示公文发文字号中的发文年份时,应使用六角括号。

\subsection*{8 书名号内用顿号表示停顿}

例:

根据《××省物价局、××省财政厅关于××市建制镇城市基础设施配套费征收标准的批复》(××规〔2012〕59号)文件要求,特制定本管理办法。(错误)

根据《××省物价局 ××省财政厅关于××市建制镇城市基础设施配套费征收标准的批复》(××规〔2012〕59号)文件要求,特制定本管理办法。(正确)

\subsection*{9 附件名称后使用标点符号}

例:

附件:1.××区查处取缔无证无照工作领导小组成员名单;(错误)

附件:1.××区查处取缔无证无照工作领导小组成员名单(正确)

解析:附件名称后不用任何标点符号。

\subsection*{10 二级标题在换行分段情况下使用句号}

例:

为减少监管环节,保证上下协调联动……(错误)



\section{标点符号的用法}

下面是十种标点符号的使用规则和常见错误,有分析、有示例,可以对照一下看自己有没有用错。

句号

一句末尾用句号,语气平缓调不高。

读书见它要停顿,作文断句莫忘掉。

基本用法

1. 用于句子末尾,表示陈述语气。

使用句号主要根据语段前后有较大停顿、带有陈述语气和语调,并不取决于句子的长短。

示例1:北京是中华人民共和国的首都。

示例2:(甲:咱们走着去吧?)乙:好。

2.有时也可表示较缓和的祈使语气和感叹语气。

示例1:请您稍等一下。

示例2:我不由地感到,这些普通劳动者同样很值得尊敬。

常见错误

1.当断不断,一逗到底。

2.不当断却断了,割裂了句子。

如:生产成本居高不下的原因,一个是设备落后,能耗高。另一个是管理不善,浪费严重。(“能耗高”后面的句号应改作逗号)

问号

有疑有问用问号,设问反问也需要。

遇它读出语调来,看书见它要思考。

基本用法

1.用于句子的末尾,表示疑问语气(包括反问、设问等疑问类型)。

使用问句主要根据语段前后有较大停顿、带有疑问语气和语调,并不取决于句子的长短。

示例1:你怎么还不回家去呢?

示例2:难道这些普通的战士不值得歌颂吗?

示例3:(一个外国人,不远万里来到中国,帮助中国的抗日战争。)这是什么精神?这是国际主义精神。

2.选择问句中,通常只在最后一个选项的末尾用问号,各个选项之间一般用逗号隔开。

当选项较短且选项之间没有停顿时,选项之间可不用逗号。当选项较多或较长,或有意突出每个选项的独立性时,也可每个选项之后都用问号。

示例1:诗中记述的这场战争究竟是真实的历史描述,还是诗人的虚构?

示例2:这是巧合还是有意安排?

示例3:要一个什么样的结尾:现实主义的?传统的?大团圆的?荒诞的?民族形式的?有象征意义的?

示例4:但到底是称赞了我什么:是有几处画得好?还是什么都敢画?抑或是一种对于失败者的无可奈何的安慰?我不得而知。

示例5:这一切都是由客观的条件造成的?还是由行为的惯性造成的?

3.在多个问句连用或表达疑问语气加重时,可叠用问号。

通常应先单用,再用叠用,最多叠用三个问号。在没有异常强烈的情感表达需要时不宜叠用问号。

示例:这就是你的做法吗?你这个总经理是怎么当的??你怎么竟敢这样欺骗消费者???

4.问号也有标号的用法,即用于句内,表示存疑或不详。

示例1:马致远(1250?—1321),大都人,元代戏曲家、散曲家。

示例2:钟嵘(?—518),颍川长社人,南朝梁代文学批评家。

示例3:出现这样的文字错误,说明作者(编者?校者?)很不认真。

常见错误

1. 句子里虽然有疑问词,但全句不是疑问句,句末却用了问号。

如:我不知道这件事是谁做的?但我猜做这件事的人一定对我们的情况比较熟悉。(问号应改作逗号)

2. 句子虽然包含选择性的疑问形式,但全句不是疑问句,句末却用了问号。

如:我也不知道你喜欢不喜欢这种颜色?(问号应改作句号)

感叹号

感情强烈句和段,其中叹号常出现。

请求反问都该用,有它文章起波澜。

基本用法

1. 用于句子的末尾,主要表示感叹语气,有时也可表示强烈的祈使语气、反问语气等。

使用叹号主要根据语段前后有较大停顿、带有感叹语气和语调或带有强烈的祈使、反问语气和语调,并不取决于句子的长短。

示例1:才一年不见,这孩子都长这么高啦!

示例2:你给我闭嘴!

示例3:谁知道他今天怎么搞的!

2. 用于拟声词后,表示声音短促或突然。

示例1:咔嚓!一道闪电划破了夜空。

示例2:咚!咚咚!传来一阵急促的敲门声。

3. 表示声音巨大或声音不断加大时,可叠用叹号;表达强烈语气时,也可叠用叹号,最多叠用三个叹号。

在没有异常强烈的情感表达需要时不宜叠用叹号。

示例1:轰!!!在这天崩地塌声音中,女娲突然醒来。

示例2:我要揭露!我要控诉!!我要以死抗争!!!

4. 当句子包含疑问、感叹两种语气且都比较强烈时(如带有强烈感情的反问句和带有惊愕语气的疑问句),可在问号后再加叹号(问号、叹号各一)。

示例1:这点困难能把我们吓到吗?!

示例2:他连这些最起码的常识都不懂,还敢说自己是高科技人才?!

常见错误

1. 滥用叹号。

陈述句末尾一般用句号,不用叹号。不能认为只要带有感情,就用叹号。

如:看到这里,他愤怒得浑身热血直往上涌!(叹号应改作句号)

2. 把句末点号叹号用在句子中间,割断了句子。

如:那优美的琴声啊!令我如痴如醉。(叹号应改作逗号)

逗号

标点符号谁最忙?逗号使用最频繁。

句子中间要停顿,往往由它来值班。

基本用法

1. 复句内各分句之间的停顿,除了有时用分号,一般都用逗号。

示例1:不是人们的意识决定人们的存在,而是人们的社会存在决定人们的意识。

示例2:学历史使人更明智,学文学使人更智慧,学数学使人更精细,学考古使人更深沉。

示例3:要是不相信我们的理论能反映现实,要是不相信我们的世界存在和谐,那就不可能有科学。

2. 用于下列的各种语法位置:

较长的主语之后

示例:苏州园林建筑各种门窗的精美设计和雕镂功夫,都令人叹为观止。

句首的状语之后

示例:在苍茫的大海上,狂风卷集着乌云。

较长的宾语之前

示例:有的考古工作者认为,南方古猿生存于上新世至更新世的初期和中期。

带句内语气词的主语(或其他成分)之后,或带句内语气词的并列成分之间

示例1:他呢,倒是很乐意地、全神贯注地干起来了。

示例2:(那是个没有月亮的夜晚。)可是整个村子——白房顶啦,白树木啦,雪堆啦,全看得见。

较长的主语之间、谓语之间、宾语之间

示例1:母亲沉痛的诉说,以及亲眼见到的事实,都启发了我幼年时期追求真理的思想。

示例2:那姑娘头戴一顶草帽,身穿一条绿色的裙子,腰间还系着一根橙色的腰带。

示例3:必须懂得,对于文化传统,既不能不分青红皂白统统抛弃,也不能不管精华糟粕全盘继承。

前置的谓语之后或后置的状语、定语之前

示例1:真美啊,这条蜿蜒的林间小路。

示例2:她吃力地站了起来,慢慢地。

示例3:我只是一个人,孤孤单单的。

3. 用于下列各种停顿处:

复指成分或插说成分前后

示例1:老张,就是原来的办公室主任,上星期已经调走了。

示例2:车,不用说,当然是头等。

语气缓和的感叹语、称谓语或呼唤语之后

示例1:哎呦,这儿,快给我揉揉。

示例2:大娘,您到哪儿去啊?

示例3:喂,你是哪个单位的?

某些序次语(“第”字头、“其”字头及“首先”类序次语)之后

示例1:为什么许多人都有长不大的感觉呢?原因有三:第一,父母总认为自己比孩子成熟;第二,父母总要以自己的标准来衡量孩子;第三,父母出于爱心而总不想让孩子在成长的过程中走弯路。

示例2:《玄秘塔碑》之所以成为书法的范本,不外乎以下几方面的因素:其一,具有楷书的点画、构体的典范性;其二,承上启下,成为唐楷的极致;其三,字如其人,爱人及字,柳公权高尚的书品、人品为后人所崇仰。

示例3:下面从三个方面讲讲语言的污染问题:首先,是特殊语言环境中的语言污染问题;其次,是滥用缩略语引起的语言污染问题;再次,是空话和废话引起的语言污染问题。

常见错误

1. 插入语没有加逗号跟其他成分分隔。

如:毫无疑问对这种人我们只能诉诸法律。(“毫无疑问”后面应加逗号)

2. 不该用逗号的地方用了逗号,把句子肢解了。

如:她暗下决心,一旦成婚,就把支持丈夫干好本职工作,作为今生今世最大的追求。(“作为”前面的逗号应去掉)

顿号

并列词语或短语,地位一样并肩站。

顿号用来做分界,读到它时停顿短。

基本用法

1. 用于并列词语之间。

示例1:这里有自由、民主、平等、开放的风气和氛围。

示例2:造型科学、技艺精湛、气韵生动,是盛唐石雕的特色。

2. 用于需要停顿的重复词语之间。

示例:他几次三番、几次三番地辩解着。

3. 用于某些序次语(不带括号的汉字数字或“天干地支”类序次语)之后。

示例1:我准备讲两个问题:一、逻辑学是什么?二、怎样学好逻辑学?

示例2:风格的具体内容主要有以下四点:甲、题材;乙、用字;丙、表达;丁、色彩。

4. 相邻或相近两数字连用表示概数,通常不用顿号。若相邻两数字连用为缩略形式,宜用顿号。

示例1:飞机在6000米高空水平飞行时,只能看到两侧八九公里和前方一二十公里范围内的地面。

示例2:这种凶猛的动物常常三五成群地外出觅食和活动。

示例3:农业是国民经济的基础,也是二、三产业的基础。

5. 标有引号的并列成分之间、标有书名号的并列成分之间通常不用顿号。若有其他成分插在并列的引号之间或并列的书名号之间(如引语或书名号之后还有括注),宜用顿号。

示例1:“日”“月”构成“明”字。

示例2:店里挂着“顾客就是上帝”“质量就是生命”的条幅。

示例3:《红楼梦》《三国演义》《西游记》《水浒传》,是我国长篇小说的四大名著。

示例4:李白的“白发三千丈”(《秋浦歌》)、“朝如青丝暮成雪”(《将进酒》)都是脍炙人口的诗句。

示例5:办公室有人订《人民日报》(海外版)、《光明日报》和《时代周刊》等报纸。

常见错误

1. 没有注意到并列词语的层次。层次不同的并列关系,上一层用逗号,次一层用顿号。

如:城市发展的近期和远景规划,包括土地的开发与利用、基础设施、生活服务设施的建设与管理、环境的治理与保护、信息的收集、处理和应用、吸引投资的网络组织、营销方式和鼓励措施等。(错误用法)

2. 词语间是包容关系而不是并列关系,中间却用了顿号。

如:新建小区内的住宅共24幢、396套,绿化率达到45% 。(中间的顿号应去掉)

3. “甚至、尤其、直至、特别是、以及、还有、包括、并且、或者”等连词前面用了顿号。

如:由于商品供求往往随着不同区域、不同季节、甚至不同客流成分的变化而变化,所以采购者应当及时把握需求信息。(“甚至”前面的顿号应改作逗号)

分号

并列句子肩并肩,不分主次紧相连。

如用逗号隔不开,可用分号站中间。

基本用法

1. 表示复句内部并列关系的分句(尤其当分句内部还有分号时)之间的停顿。

示例1:语言文字的学习,就理解方面说,是得到一种知识;就运用方面说,是养成一种习惯。

示例2:内容有分量,尽管文章短小,也是有分量的;内容没有分量,即使写得再长也没有用。

2. 表示非并列关系的多重复句第一层(主要是选择、转折等关系)之间的停顿。

示例1:人还没看见,已经先听见歌声了;或者人已经转过山头望不见了,歌声还余音袅袅。

示例2:尽管人民革命的力量在开始时是弱小的,所以总是受压迫的;但是由于革命的力量代表历史发展的方向,因此本质上又是不可战胜的。

示例3:不管一个人如何伟大,也总是生活在一定的环境和条件下;因此个人的见解总难免带有某种局限性。

示例4:昨天夜里下了一场雨,以为可以凉快些;谁知没有凉快下来,反而更热了。

3. 用于分项列举的各项之间。

示例:特聘教授的岗位职责:一、讲授本学科的主干基础课程;二、主持本学科的重大科研项目;三、领导本学科的学术队伍建设;四、带领本学科赶超或保持世界先进水平。

常见错误

1. 单句内并列词语之间用了分号。

如:报名者请携带户口簿;身份证;高中毕业证书;体检证明;两张二寸近期免冠照片。(四个分号都应改作逗号)

2. 不是并列关系就不能用分号。

如:这些展品不仅代表了两千多年前我国养蚕、纺织、印染、刺绣和缝纫工艺方面所达到的高度水平;而且也显示了我国古代劳动人民的聪明智慧和创造才能。(“而且”前面的分号应改作逗号)

3. 多重复句中,并列的分句不是处在第一层上,之间却用了分号。

如:只有健全社会主义法制,才能使社会主义民主法律化、制度化;才能用法律手段管理经济;才能维护安定团结的政治局面,保障社会主义现代化建设的顺利进行。(“经济”后面的分号应改作逗号)

4. 被分号分隔的语句内出现了句号。

须知:分号所表示的停顿或分隔的层次小于句号。

冒号

小小冒号两个点,提示下文常出现。

它和引号是朋友,文章之中常相伴。

基本用法

1. 用于总说性或提示性词语(“说”“例如”“证明”)之后,表示提示下文的。

示例1:北京紫禁城有四座城门:午门、神武门、东华门和西华门。

示例2:他高兴地说:“咱们去好好地庆祝一下吧!”

示例3:小王笑着点了点头:“我就是这么想的。”

示例4:这一事实证明:人能创造环境,环境同样也能创造人。

2. 表示总结上文。

示例:张华上了大学,李萍进了技校,我当了工人:我们都有美好的前途。

3. 用在需要说明的词语之后,表示注释和说明。

示例1:(本市将举办首届大型书市。)主办单位:市文化局;承办单位:市图书进口公司;时间:8月15日—20日;地点:市体育馆观众休息厅。

示例2:(做阅读理解题有两个办法。)办法之一:先读题干,再读原文,带着问题有针对性地读课文。办法之二:直接读原文,读完再做题,减少先入为主的干扰。

4. 用于书信、讲话稿中称谓语或称呼语之后。

示例1:广平先生:……

示例2:女士们、先生们:……

5. 一个句子内部一般不应套用冒号。在列举式或条纹式表述中,如不得不套用冒号时宜另起段落来显示各个层次。

示例:第十条 遗产按照下列顺序继承:第一顺序:配偶、子女、父母。第二顺序:兄弟姐妹、祖父母、外祖父母。

常见错误

1. 冒号套用。应避免一个冒号范围里再用冒号。

如:心理学研究表明:影响儿童心理发展有三个重要因素:遗传、环境和教育。(第一个冒号应改作逗号)

2. 提示性动词指向引文之后的词语,这个动词之后却用了冒号。

如:厂领导及时提出:“以强化管理抓节约挖潜、以全方位节约促成本降低、以高质量低成本开拓市场增效益”的新思路。(句中的冒号应去掉)

3. 冒号用在了没有停顿的地方。

如:女乘务员小心地端起杯子,正准备换个地方放,突然,随着一声:“谁让你动我的杯子”的怒吼,一位30多岁的年轻人,一把夺走了杯子。(句中的冒号应去掉)

4. 冒号与“即”“也就是”一类的词语同时使用。

如:他们加强了施工现场每一个环节、每一道工序甚至每一个工点的安全管理。对于施工中出现的安全事故苗头实行“三不放过”:即没查出原因不放过,当事人和施工人员没有深刻认识事故苗头的后果不放过,整改措施没有落实不放过。(句中的冒号应改作逗号,或者保留冒号去掉“即”字)

引号

四个蝌蚪真奇妙,前揽后抱是引号。

人物语言引在内,别人文句用它标。

基本用法

1. 标示语段中直接引用的内容。

示例:李白诗中就有“白发三千丈”这样极尽夸张的语句。

2. 表示需要着重论述或需要强调的内容。

示例:这里所谓的“文”,并不是指文字,而是指文采。

3. 表示语段中具有特殊含义而需要特别指出的成分,如别称、简称、反语等。

示例1:电视被称作“第九艺术”。

示例2:人类学上常把古人化石尼安德特人,简称“尼人”。

示例3:有几个“慈祥”的老板把捡来的菜叶用盐浸浸就算作工友的菜肴。

4. 一层用双引号,里面一层用单引号。

示例:他问:“老师,‘七月流火’是什么意思?”

5. 独立成段的引文如果只有一段,段首和段尾都用引号;不止一段时,每段开头仅用前引号,只在最后一段末尾用后引号。

示例1:我曾在报纸上看到这样谈幸福:“幸福是知道自己喜欢什么和不喜欢什么……幸福是知道自己擅长什么和不擅长什么……幸福是在正确的时间做出了正确的选择……”

6. 在书写带月、日的事件、节日或其他特定意义的短语(含简称)时,通常只标引其中的月和日;需要突出和强调该事件或节日本身时,也可连同事件和节日一起标引。

示例1:“5·12”汶川大地震。

示例2:“五四”以来的话剧,是我国戏剧中的新形式。

示例3:纪年“五四运动”90周年。

常见错误

1. 滥用引号。词语没有特殊含义,随便加上了引号。

如:樱花飘落时,就像漫天的“雪花”在飞舞。(句中的引号应去掉)

2. 引号前后相关的标点处理错误。

如:常言说得好“无酒不成宴”,酒的选择非常关键,因为它最能调动人的激情。

(可改作:常言说得好,“无酒不成宴”。酒的选择非常关键,因为它最能调动人的激情。常言说得好:“无酒不成宴。”酒的选择非常关键,因为它最能调动人的激情。)

省略号

省略号,六个点,千言万语全包揽。

表示省略用到它,说话断续把它添。

基本用法

1. 标示引文的省略。

示例:我们齐声朗诵起来:“……俱往矣,数风流人物,还看今朝。”

2. 标示列举或重复词语的省略。

示例:对政治的敏感,对生活的敏感,对性格的敏感……这都是作家必须要有的素质。

3. 标示语意未尽。

示例3:在人迹罕至的深山密林里,假如突然看见一缕炊烟……

示例4:你这样干,未免太……!

4. 标示说话时断断续续。

示例:她磕磕巴巴地说:“可是……太太……我不知道……你一定是认错人了。”

5. 标示对话中的沉默不语。

示例:“还没结婚吧?”

“……”他飞红了脸,更加忸怩起来。

6. 标示特定的成分虚缺。

示例:只要……就……

7. 在标示诗行、段落的省略时,可连用两个省略号(即相当于十二连点)。

常见错误

1. 滥用省略号。

如:为什么街头小青年满口脏字?为什么摩登女郎徒有其表,一张口就是污言秽语……?(应去掉省略号)

2. 省略号和“等”“之类”并用。因为省略号的作用相当于“等”“等等”“之类”。两者不能并用。

如:在另一领域中,人却超越了自然力,如飞机、火箭、电视、计算机……等等。(应去掉省略号)

书名号

书名号,前后弯,标明书籍和报刊。

篇名曲名也可用,标得清楚方便看。

基本用法

1. 标示书名、卷名、篇名、刊物名、报纸名、文件名等。

示例1:《红楼梦》(书名)

示例2:《史记·项羽本纪》(卷名)

示例3:《论雷峰塔的倒掉》(篇名)

示例4:《每周关注》(刊物名)

示例5:《人民日报》(报纸名)

示例6:《全国农村工作会议纪要》(文件名)

2. 标示电影、电视、音乐、诗歌、雕塑等各类用文字、声音、图像等表现的作品的名称。

示例1:《渔光曲》(电影名)

示例2:《追梦录》(电视剧名)

示例3:《勿忘我》(歌曲名)

示例4:《沁园春·雪》(诗词名)

示例5:《东方欲晓》(雕塑名)

示例6:《光与影》(电视节目名)

示例7:《社会广角镜》(栏目名)

示例8:《庄子研究文献数据库》(光盘名)

示例9:《植物生理学系列挂图》(图片名)

3. 标示全中文或中文在名称中占主导地位的软件名。

示例:科研人员正在研制《电脑卫士》杀毒软件。

4. 标示作品名的简称。

示例:我读了《念青唐古拉山脉纪行》一文(以下简称《念》),收获很大。

5. 当书名号中还需要用书名号时,里面一层用单书名号,外面一层用双书名号。

示例:《教育部关于提请审议的报告》

常见错误

滥用书名号,随意超出应用范围。

如:品牌名、证件名、会议名、展览名、奖状名、奖杯名、活动名、机构名,也用书名号标示。


